\documentclass[]{article}

\usepackage[utf8]{inputenc}

\usepackage[T1]{fontenc}

\usepackage[french]{babel}

\usepackage{amsmath, amsfonts, amssymb, amsthm}

\usepackage{fullpage}

\usepackage{enumerate}
\usepackage{graphicx}
\usepackage{algorithm}
\usepackage{algorithmic}

\usepackage{hyperref}
\hypersetup{
    colorlinks,
    citecolor=black,
    filecolor=black,
    linkcolor=black,
    urlcolor=black
}


%Pour les algos
\floatname{algorithm}{Algorithme}
\renewcommand{\algorithmicrequire}{\textbf{Entrée:}}
\renewcommand{\algorithmicensure}{\textbf{Sortie:}}
\renewcommand{\algorithmicif}{\textbf{si}}
\renewcommand{\algorithmicthen}{\textbf{alors}}
\renewcommand{\algorithmicelse}{\textbf{sinon}}

\title{
{\Huge Page Title}\\
Cursus name\\
}

\author{
\textbf{Dell’Aria Doriano}\\
}


\date{\today\\
Année Académique 2020-2021\\
Bachelier en Sciences Informatiques\\
\vspace{1cm}
Faculté des Sciences, Université de Mons}

\begin{document}

\maketitle
\pagebreak

% \tableofcontents
% \pagebreak

    \section{Section 1}
    text

% ====== exemple d'algo =======
% \begin{algorithm}
%     \caption{Recherche linéaire du maximum}
%     \begin{algorithmic}[1]
%     \REQUIRE un tableau d’entiers $A$
%     \ENSURE la valeur du plus grand entier contenu dans $A$
%     \STATE $max \leftarrow -\infty$
%     \FOR{$i \leftarrow 1$ `a $longueur[A]$}
%     \IF{$max < A[i]$}
%     \STATE $max \leftarrow A[i]$
%     \ENDIF
%     \ENDFOR
%     \RETURN $max$
%     \end{algorithmic}
% \end{algorithm}


\end{document}